\newpage
\subsection{TSEngine Internal Core Abstraction}
\hspace{\parindent} Internal TSEngine Core covers \hyperref[sec:renderer]{\ref*{sec:renderer} Renderer} and \hyperref[sec:headset]{\ref*{sec:headset} Headset with Controllers} abstraction, there is one place where these two things share the same encapsulation, this thing is \texttt{Context} class.\\
All of them are available in the corresponding files \texttt{*.cpp} and \texttt{*.h} files inside \texttt{./engine/src/core/} path]\\
The main purpose of \texttt{Context} class is creation of OpenXR and Vulkan instances. So it means that Khronos structures are initialized within this class, when usually there is no reason to re-initialized them after creation of the app and used widely through the app, such structures objects are: \texttt{VkPhysicalDevice}, \texttt{VkDevice}, \texttt{VkQueue}, or \texttt{XrSystemId}. And also \texttt{Context} sets up multisampling and uniform buffer alignment. This is not the end as \texttt{Context} creates and sets up debug messengers.
It's worth to point out that Context also loading Vulkan functions with the help of \hyperref[sec:vkLoader]{\ref*{sec:vkLoader} Vulkan Loader}, unfortunately we didn't have time to provide our own OpenXR loader.\\
I see here a field of improvement as it can be accessible through singleton pattern
\footnote{The Singleton Pattern is a design pattern used in software engineering to ensure that a class has only one instance and provides a global point of access to it.}.
\subsubsection{Renderer}
\label{sec:renderer}
\hspace{\parindent} We can distinguish the front-end of the renderer being the part of ECS - \hyperref[sec:renderer_system]{\ref*{sec:renderer_system}. ECS Renderer System}, but also back-end in classes:
\begin{itemize}
    \item \texttt{DataBuffer}\\
    It encapsulates Vulkan buffer handling, abstracting details such as buffer creation, memory management, and data transfer.
    \item \texttt{ImageBuffer}\\
    It is designed for encapsulation of the creation and management of Vulkan images, device memory allocations, and image views,
    \item \texttt{Pipeline}\\
    This class encompass the complexity of Vulkan many different stages and settings, from shader stages to fixed-function state like blending, depth testing, providing a more streamlined interface for setting up and using these pipelines in rendering operations. 
    \item \texttt{RenderTarget}
    It is designed to encapsulate the concept of a render target in Vulkan. It manages the creation and lifecycle of a framebuffer and associated image views, which are fundamental to rendering operations. The framebuffer combines color and depth/stencil attachments, which are used during a render pass. 
    \item \texttt{RendererProcess}
    It manages uniform data for objects and lights, handles synchronization primitives, and maintains various Vulkan objects essential for rendering operations.
    \item \texttt{Renderer}
    Comprehensive component that manages various aspects of the rendering pipeline in a Vulkan environment that plays a role of back-end of the renderer. It orchestrates render processes, manages Vulkan resources like command pools and descriptor pools, and handles different pipelines for rendering. The class's structure emphasizes controlled resource management, synchronization, and the ability to handle multiple frames in flight, which are key elements in high-performance graphics applications.
\end{itemize}
\subsubsection{Headset with Controllers}
\label{sec:headset}
OpenXR specific staff is handled by classes:
\begin{itemize}
    \item \texttt{Headset}\\
    \item \texttt{MirrorView}\\
    \item \texttt{Controllers}\\
    Controllers abstraction provides a basic layer of abstraction over OpenXR control devices. 
\end{itemize}