\newpage

\subsection{Tests}
Testing of software is very important nowadays with long-living software, an example of such long-living software are graphics engines.
One of the simplest methods of testing the software are unit tests, this is how we implemented testing our engine.

The interface for testing the engine:
\subsubsection{Engine Tester Header}
\begin{lstlisting}[language=c++, caption=Tester Engine class (./engine/tests/tester.cpp)]
struct TesterEngine : public Engine
{
    TesterEngine(std::chrono::steady_clock::duration renderingDuration_) : renderingDuration{renderingDuration_}
    {}

    TesterEngine(const TesterEngine&) = delete;
    TesterEngine& operator=(TesterEngine&) = delete;
    TesterEngine(const TesterEngine&&) = delete;
    TesterEngine& operator=(TesterEngine&&) = delete;

    virtual bool tick(const float dt) override { return true; }
    virtual void loadLvL() override
    {}

    std::chrono::steady_clock::duration renderingDuration;
};
\end{lstlisting}


Testing also requires a few inserts to the engine main loop. Tester Adapter code is inserted only if \texttt{TESTER\_ADAPTER} preprocessor definition is defined, it helps to avoid unnecessary operation in shippable product. Testing is turned on at the level of project configuration - more about it in \hyperref[sec:build_unit_tests]{Bulding Unit Tests} section. 
\begin{lstlisting}[language=c++, caption=Tester related code inside Engine Main Loop (./engine/tests/tester.cpp)]
    \\ Before entering the main loop
    const auto testerAdapter = dynamic_cast<TesterEngine*>(game);
    bool isRenderingStarted{};

    ...

    \\ Inside the main loop
    if ((testerAdapter != nullptr) && isRenderingStarted)
    {
        if (std::chrono::steady_clock::now() >= (startTime + testerAdapter->renderingDuration))
        {
            loop = false;
        }
    }

    ...

    if (!isRenderingStarted)
    {
        isRenderingStarted = true;
    }
\end{lstlisting}

\subsubsection{Tests}
The list of tests which are performed:
\begin{itemize}
    \item DummyTests
        \begin{itemize}
            \item DummyTest This test should be always true, checks only if testing functionality works.
            \begin{lstlisting}[language=c++, caption=]
TEST(DummyTests, Dummytest)
{
    const auto multiplication = 28 * 10 * 2000;
    const auto expected = 560000;

    ASSERT_EQ(expected, multiplication);
}
            \end{lstlisting}
        \end{itemize}
    \item MathTests Tests \hyperref[]{}
        \begin{itemize}
            \item mat4MultiplicationTest Verifies if 4x4 matrix multiplication works correctly.
            \begin{lstlisting}[language=c++, caption=]
TEST(MathTests, mat4MultiplicationTest)
{
    const ts::math::Mat4 leftMatrix
    {
        1, 2, 3, 4,
        1, 2, 3, 4,
        5, 6, 7, 8,
        5, 6, 7, 8,
    };

    const ts::math::Mat4 rightMatrix
    {
        1, 2, 3, 4,
        1, 2, 3, 4,
        5, 6, 7, 8,
        5, 6, 7, 8,
    };

    const ts::math::Mat4 expected
    {
        38, 48 , 58 , 68 ,
        38, 48 , 58 , 68 ,
        86, 112, 138, 164,
        86, 112, 138, 164,
    };

    const auto multiplication = rightMatrix * leftMatrix;
    ASSERT_EQ(ts::math::to_string(expected), ts::math::to_string(multiplication));
}
            \end{lstlisting}
            \item mat3MultiplicationTest Verifies if 3x3 matrix multiplication works correctly.
            \begin{lstlisting}[language=c++, caption=]
TEST(MathTests, mat3MultiplicationTest)
{
    const ts::math::Mat3 leftMatrix
    {
        0.4f, 0.5f, 0.6f,
        0.7f, 0.8f, 0.9f,
        0.1f, 0.2f, 0.3f,
    };

    const ts::math::Mat3 rightMatrix
    {
        0.11f, 0.12f, 0.13f,
        0.14f, 0.15f, 0.16f,
        0.17f, 0.18f, 0.19f,
    };

    const ts::math::Mat3 expected
    {
        0.216f, 0.231f, 0.246f,
        0.342f, 0.366f, 0.39f ,
        0.09f , 0.096f, 0.102f,
    };

    const auto multiplication = rightMatrix * leftMatrix;
    ASSERT_EQ(ts::math::to_string(expected), ts::math::to_string(multiplication));
}
            \end{lstlisting}
            \item mat2InversionTest Verifies if inversion of 2x2 matrix works correctly.
            \begin{lstlisting}[language=c++, caption=]
TEST(MathTests, mat2InversionTest)
{
    const ts::math::Mat2 matrix
    {
        1.f, 3.f,
        4.f, 2.f,
    };

    const ts::math::Mat2 identityMatrix
    {
        1.f, 0.f,
        0.f, 1.f,
    };

    const auto invertedMatrix = ts::math::inverse(matrix);
    const auto s = matrix * invertedMatrix;
    ASSERT_EQ(ts::math::to_string(identityMatrix), ts::math::to_string(s));
}
            \end{lstlisting}
            \item mat3InversionTest Verifies if inversion of 3x3 matrix works correctly.
            \begin{lstlisting}[language=c++, caption=]
TEST(MathTests, mat3InversionTest)
{
    const ts::math::Mat3 matrix
    {
        1.f, 2.f, 3.f,
        0.f, 1.f, 4.f,
        5.f, 6.f, 0.f,
    };

    const ts::math::Mat3 expected
    {
        -24.f, +18.f, +5.f,
        +20.f, -15.f, -4.f,
        -5.f , +4.f , +1.f,
    };

    const auto invertedMatrix = ts::math::inverse(matrix);
    ASSERT_EQ(ts::math::to_string(invertedMatrix), ts::math::to_string(expected));
}
            \end{lstlisting}
            \item mat4InversionTest Verifies if inversion of 4x4 matrix works correctly.
            \begin{lstlisting}[language=c++, caption=]
TEST(MathTests, mat4InversionTest)
{
    const ts::math::Mat4 matrix
    {
        3.f, 4.f, 2.f, 1.f,
        0.f, 1.f, 0.f, 0.f,
        2.f, 1.f, 3.f, 2.f,
        1.f, 0.f, 1.f, 1.f,
    };

    const ts::math::Mat4 expected
    {
        +0.5f, -1.5f, -0.5f, +0.5f,
        -0.f , +1.f , -0.f , +0.f ,
        +0.f , -1.f , +1.f , -2.f ,
        -0.5f, +2.5f, -0.5f, +2.5f,
    };

    const auto invertedMatrix = ts::math::inverse(matrix);
    ASSERT_EQ(ts::math::to_string(invertedMatrix), ts::math::to_string(expected));
}
            \end{lstlisting}
            \item mat4rotationTest  Verifies if rotation functionality works correctly.
            \begin{lstlisting}[language=c++, caption=]
TEST(MathTests, mat4rotationTest)
{
    const auto matrix = ts::math::Mat4(1.f);

    const ts::math::Vec3 axis{0.f, 1.f, 0.f};
    const auto result = ts::math::rotate(matrix, axis, ts::math::radians(180.f));

    const ts::math::Mat4 expected
    {
        -1.f, +0.f, +0.f, +0.f,
        +0.f, +1.f, +0.f, +0.f,
        +0.f, +0.f, -1.f, +0.f,
        +0.f, +0.f, +0.f, +1.f,
    };
    ASSERT_TRUE(expected[0].x == result[0].x and expected[1].y == result[1].y and expected[2].z == result[2].z);
}
            \end{lstlisting}
        \end{itemize}
    \item TestGame
        \begin{lstlisting}[language=c++, caption=]
class TestGame final : public ts::TesterEngine
{
    static constexpr std::chrono::steady_clock::duration renderingDuration{3s};

public:
    TestGame() : TesterEngine{renderingDuration}
    {}
};
        \end{lstlisting}
    \item GameTests
        \begin{itemize}
            \item IsThrowsException
            \begin{lstlisting}[language=c++, caption=]
TEST(GameTests, IsThrowsException)
{
    try
    {
        const auto game = std::make_unique<TestGame>();
        const auto result = ts::run(game.get());
    }
    catch (...)
    {
        FAIL();
    }
}
            \end{lstlisting}
            \item RunReturnsZero
            \begin{lstlisting}[language=c++, caption=]
TEST(GameTests, RunReturnsZero)
{
    const auto game = std::make_unique<TestGame>();
    const auto result = ts::run(game.get());

    ASSERT_EQ(0, result);
}
            \end{lstlisting}
        \end{itemize}
\end{itemize}

More about software development and methods of testing are available in \hyperref[sec:testing]{TODO} section.