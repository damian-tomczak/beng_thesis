\newpage
\subsection{Operating System Specific}
\label{sec:os}
\hspace{\parindent}
So far, engine supports only Windows system. The main reason for this is poor support of VR devices on Linux systems.
It is further explained in the section \hyperref[sec:problems]{\ref*{sec:problems} Problems During the Development}.
Every OS specific implementations should be specified inside \texttt{./engine/src/os/[OS API TYPE]/} with declarations at \texttt{./engine/src/os/[OS API TYPE]/os.h}.

Only \texttt{.cpp} files used in active operating system are chosen for the compilation, the same applies to headers, especially \texttt{os.h} which existence is required and from which the engine takes OS specific stuff.
\begin{lstlisting}[language=c++, caption=Windows OS Header (./engine/src/os/win32/os.h)]
#define LIBRARY_TYPE HMODULE
#define LoadFunction GetProcAddress

class Win32Window final : public Window
{
public:
    Win32Window(const std::string_view windowName, const size_t width, const size_t height)
        : Window{windowName, width, height}
    {}
    ~Win32Window();

    HINSTANCE getHInstance() const { return mHInstance; }
    HWND getHwnd() const { return mHwnd; }

    void show() override;
    Message peekMessage() override;
    void dispatchMessage() override;

private:
    static LRESULT windowProcedure(HWND pHwnd, UINT msg, WPARAM pWParam, LPARAM wLParam);

    HINSTANCE mHInstance{};
    HWND mHwnd{};
    MSG mMsg{};

    void createWindow() override;
};
\end{lstlisting}

So far, in the \texttt{os.h} header, for engine to work, it is obligatory to define \texttt{LIBRARY\_TYPE} and \texttt{LoadFunction} preprocessor definitions for \hyperref[sec:vkLoader]{\ref*{sec:vkLoader} Vulkan Loader}, and \texttt{Window} class that creates window and handles user actions using interface's methods, in the case of Windows using Win32 API.

Unfortunately, it's not possible to fit everything in one header file, therefore for an instance, object of \texttt{Window} class must be created inside 
\texttt{Window::createWindowInstance} function.
However, despite testing only one OS, we provided an architecture which is capable of being used on other platforms, which is explained in the
\hyperref[sec:build_os]{\ref*{sec:build_os} Building OS Specific} section.