\newpage

\subsection{Example Game}
The starting project:
\begin{verbatim}
└───game
    ├───components
    │    ├───echo_component.hpp
    │    └───example_component.hpp
    ├───systems
    │    └───example_system.hpp
    ├───CMakeLists.txt
    ├───game.cpp
    └───game.h
\end{verbatim}
\begin{table}[h]
\caption{Structure of the starting project}
\end{table}
Starting project is consisted of two directories that are the part of ECS system: \texttt{components} and \texttt{systems}.\\Elaboration about the content of them can be found in \hyperref[sec:game_components]{Game Components} section.\\ The heart of game is placed in \texttt{game.cpp} and \texttt{game.h} files. The last item to mention is \texttt{CMakeLists.txt} file that is the part of \hyperref[sec:build]{build system}.

To make possible to run the process, a class that implements \hyperref[sec:engine_class]{\texttt{Engine} interface} is mandatory. 
\begin{lstlisting}[language=c++, caption=Example of Game class header (./game/game.h)]
class Game final : public ts::Engine
{
public:
    Game() = default;

    bool init(const char*& gameName, unsigned& width, unsigned& height) override;
    void loadLvL() override;
    bool tick(const float dt) override;
    void close() override;
};
\end{lstlisting}
To then create an instance of Game class and \hyperref[sec:run_fun]{run the engine}.
\begin{lstlisting}[language=c++, caption=Game instance (./game/game.cpp)]
TS_MAIN()
\end{lstlisting}
Default game's \texttt{init} function sets only game name used further as the part of \hyperref[sec:window]{window name} and passed to GPU's driver.
\begin{lstlisting}[language=c++, caption=Example of Game Init Function (./game/game.cpp)]
bool Game::init(const char*& gameName, unsigned&, unsigned&)
{
    gameName = GAME_NAME;

    return true;
}
\end{lstlisting}
\texttt{loadLvL} method sets up the scene of the game, which means adding systems, creation of entities with binding to them needed components with starting values and tagging them and grouping, that's all is the part of \hyperref[sec:ecs]{ECS}.\\ The followed code snippet adds one system to the \texttt{MainRegistry} - \texttt{ExampleSystem}. Creates two  entities with \texttt{MeshComponent}: \texttt{village} and \texttt{polonez}. Moreover, creates three spheres with \hyperref[sec:pbr]{PBR renderer pipeline}, and two light sources.
\begin{lstlisting}[language=c++, caption=Example of Game Load Level function (./game/game.cpp)]
void Game::loadLvL()
{
    ts::getMainReg().addSystem<ExampleSystem>();

    auto village = ts::getMainReg().createEntity();
    village.setTag("village");
    village.addComponent<ts::TransformComponent>();
    village.addComponent<ts::RendererComponent<ts::PipelineType::NORMAL_LIGHTING>>();
    village.addComponent<ts::MeshComponent>("assets/models/village.obj");

    auto polonez = ts::getMainReg().createEntity();
    polonez.setTag("polonez");
    polonez.addComponent<ts::TransformComponent>(ts::math::Vec3{0.f, 0.f, -10.f});
    polonez.addComponent<ts::RendererComponent<ts::PipelineType::NORMAL_LIGHTING>>();
    polonez.addComponent<ts::MeshComponent>("assets/models/polonez.obj");
    polonez.addComponent<EchoComponent>("vroom vroom");

    static constexpr size_t spheresNumber = 3;

    for (size_t i{}; i < spheresNumber; ++i)
    {
        auto sphere = ts::getMainReg().createEntity();
        sphere.setTag("sphere" + std::to_string(i));
        sphere.group("spheres");
        const ts::math::Vec3 pos{spheresNumber / 3 * -5.f + 5.f * i, 2.f, -5.f};
        sphere.addComponent<ts::TransformComponent>(pos);
     
        const auto material = ts::RendererComponent<ts::PipelineType::PBR>::Material::create(
            ts::RendererComponent<ts::PipelineType::PBR>::Material::Type::GOLD);

        sphere.addComponent<ts::RendererComponent<ts::PipelineType::PBR>>(material);
        sphere.addComponent<ts::MeshComponent>("assets/models/sphere.obj");
        const auto endPos = pos + ts::math::Vec3{0, 0, -50};
        sphere.addComponent<ExampleComponent>(pos, endPos);
    }

    auto light1 = ts::getMainReg().createEntity();
    light1.setTag("light1");
    light1.addComponent<ts::TransformComponent>(ts::math::Vec3{0.f, 5.f, -7.f});
    light1.addComponent<ts::RendererComponent<ts::PipelineType::LIGHT>>();

    auto light2 = ts::getMainReg().createEntity();
    light2.setTag("light2");
    light2.addComponent<ts::TransformComponent>(ts::math::Vec3{0.f, 5.f,  0.f});
    light2.addComponent<ts::RendererComponent<ts::PipelineType::LIGHT>>();
}
\end{lstlisting}
In the default game's tick function, there is no too much to explain - \texttt{ExampleSystem} is updated.
\begin{lstlisting}[language=c++, caption=Example of Game Tick Function (./game/game.cpp)]
bool Game::tick(const float dt)
{
    ts::getMainReg().getSystem<ExampleSystem>().update();

    return true;
}
\end{lstlisting}
At the end of the default game, a goodbye message is displayed:
\begin{lstlisting}[language=c++, caption=Example of Game Close Function (./game/game.cpp)]
void Game::close()
{
    TS_LOG("Thanks for playing!");
}
\end{lstlisting}