\newpage
\section{Code Description}
\label{sec:code_descr}
\hspace{\parindent}
This code description tries to present as much useful knowledge based on code snippets as possible, therefore uses simplifications where it is needed and tries to give more architecture overview, and showing features available rather than focusing on explaining every possible implementation. All of this, paired with the fact that it is written in the spirit of documentation of TSEngine, leads to the fact that you can find this fragment of thesis much more link oriented.

We use \texttt{CamelCase} in C++ code and Allman indentation convention unless code is strictly related to C++'s STL then we use \texttt{Snake\_Case}, this happens mostly while touching C++ 20's Concepts feature.
Furthermore, \texttt{Snake\_Case} convention is also used for naming two/n-part named directories and files.

We have principles we followed in the aspect of OOP while working with \hyperref[sec:khronos]{\ref*{sec:khronos}. Khronos APIs} according to RAII\footnote{"Resource Acquisition Is Initialization," is a programming idiom used in C++ and other languages to manage resources such as memory, file handles, and network connections. The core idea behind RAII is that resource allocation (or acquisition) is tied to the lifespan of an object, ensuring automatic resource deallocation (or release) when the object is destroyed.} concepts. To achieve it, we were avoiding creating Vulkan or OpenXR objects in constructors of our abstraction, but remembering to set them as null pointers, so that we have an option to check if they are not null and if they are, destroy Khronos's objects.

About usage of \hyperref[sec:khronos]{\ref*{sec:khronos} Khronos's APIs} we will be economical with words because as it is stated at the beginning of this chapter, that our thesis is oriented around the architecture of our code, rather than explaining the API. In the case of Khronos's APIs it would be a tremendous effort and quite useless because on the market, there are available works which much better exhaust the topic of Vulkan and OpenXR - refer to the corresponding sections: \hyperref[sec:stack_vk]{\ref*{sec:stack_vk} Vulkan} and \hyperref[sec:stack_xr]{\ref*{sec:stack_xr} OpenXR}.

\newpage
\subsection{Directories Structure}
The structure of directories used in the project:
\begin{verbatim}
├───.github
│   └───workflows
├───assets
│   ├───models
│   └───shaders
├───engine
│   ├───include
│   │   └───tsengine
│   │       └───ecs
│   │           └───components
│   ├───src
│   │   ├───core
│   │   ├───ecs
│   │   │   └───systems
│   │   ├───os
│   │   │   └───win32
│   │   └───vulkan_tools
│   └───tests
├───external
│   ├───glslang
│   ├───googletest
│   ├───openxr
│   ├───tinyobjloader
│   └───vulkan
└───game
    ├───components
    └───systems
\end{verbatim}
\begin{table}[h]
\caption{Structure of directories}
\end{table}

\subsection{Namespaces}
\label{sec:namespaces}
\hspace{\parindent}
A few namespaces can be distinguished in the engine:
\begin{itemize}
    \item ts\\
    Main engine's namespace - all the things exposed to the user is available through this namespace.
    \begin{itemize}
        \item khronos\_utils\\
            Namespace containing code useful while working with OpenXr and Vulkan APIs.
        \item vkLoader\\
            Inside vkLoader the loader of Vulkan is available.
        \item Version\\
            Namespace defined in the build system that takes form of:
\begin{lstlisting}[language=c++, caption= Definition of TS\_VER preprocessor macro (.engine/CMakeLists.txt)]
TS_VER=v${PROJECT_VERSION_MAJOR}_${PROJECT_VERSION_MINOR}_${PROJECT_VERSION_PATCH}
\end{lstlisting}
            The purpose of this macro is to prevent the ABI of the engine from breaking - \hyperref[sec:build_abi]{\ref*{sec:build_abi} Common ABI}.
    \end{itemize}
\end{itemize}