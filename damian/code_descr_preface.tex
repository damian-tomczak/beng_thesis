\newpage
\section{Code Description}
\label{sec:code_descr}
This code description tries to present as much as possible useful knowledge basing on code snippets, therefore uses simplifications where it is needed and tries to give more architecture overview, features available than focusing on explanation about every possible implementation. Moreover, sometimes deliberations about lack of features or especially what could be done better can be found.
\subsection{Naming Conventions}
We use Camel Case in C++ code and Allman indentation convention unless code is strictly related to C++'s STL then we use Snake\_Case, this happens among others while touching C++ 20's Concepts feature.
Furthermore, Snake\_Case convention we use also for naming two/n-part named directories and files.

\subsection{Directories Structure}
The structure of directories used in the project:
\begin{verbatim}
├───.github
│   └───workflows
├───assets
│   ├───models
│   └───shaders
├───engine
│   ├───include
│   │   └───tsengine
│   │       └───ecs
│   │           └───components
│   ├───src
│   │   ├───core
│   │   ├───ecs
│   │   │   └───systems
│   │   ├───os
│   │   │   └───win32
│   │   └───vulkan_tools
│   └───tests
├───external
│   ├───glslang
│   ├───googletest
│   ├───openxr
│   ├───tinyobjloader
│   └───vulkan
└───game
    ├───components
    └───systems
\end{verbatim}
\begin{table}[h]
\caption{Structure of directories}
\end{table}

\subsection{Namespaces}
\label{sec:namespaces}
We can distinguish in the engine a few namespaces:
\begin{itemize}
    \item ts\\
    Main engine's namespace - all the staff exposed to the user is available through this namespace.
    \begin{itemize}
        \item khronos\_utils\\
            Namespace containing staff, useful while working with OpenXr and Vulkan APIs
        \item vkLoader\\
            Inside vkLoader is available loader of Vulkan.
        \item Version\\
            Namespace defined in the build system that takes form of:
\begin{lstlisting}[language=c++, caption= Definition of TS\_VER preprocessor macro (.engine/CMakeLists.txt)]
TS_VER=v${PROJECT_VERSION_MAJOR}_${PROJECT_VERSION_MINOR}_${PROJECT_VERSION_PATCH}
\end{lstlisting}
            The purpose of this macro is to prevent from breaking the API of the engine.
    \end{itemize}
\end{itemize}