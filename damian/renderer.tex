\newpage
\subsection{Renderer}
\label{sec:renderer}
We can distinguish the front-end of the renderer being the part of ECS - \hyperref[sec:renderer_system]{\ref*{sec:renderer_system}. ECS Renderer System}, but also back-end in classes:
\begin{itemize}
    \item \texttt{Context}\\
    The main purpose of \texttt{Context} class is creation of \hyperref[sec:openxr]{\ref*{sec:openxr}} and Vulkan instances. Besides of it, other Vulkan's structures are initialized within this class, when usually there is no reason to re-initialized them after creation of the app and used widely through the app, such structures objects are: \texttt{VkPhysicalDevice}, \texttt{VkDevice}, \texttt{VkQueue}. I see here a field of improvement as it can be through singleton pattern \footnote{The Singleton Pattern is a design pattern used in software engineering to ensure that a class has only one instance and provides a global point of access to it.}.
    \item \texttt{DataBuffer}
    \item \texttt{ImageBuffer}
    \item \texttt{Pipeline}
    \item \texttt{RenderTarget}
    \item \texttt{RendererProcess}
    \item \texttt{Renderer}
\end{itemize}
All of them are available in the corresponding files \texttt{*.cpp} and \texttt{*.h} files inside \texttt{./engine/src/core/} path]