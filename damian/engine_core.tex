\newpage
\subsection{Engine Core}
\label{sec:engine_core}
\hspace{\parindent} When talking about TSEngine Core, we distinguish public and private part. The first one is visible for a game and the latter is not.
\subsubsection{Public TSEngine Core}
\label{sec:public_engine}
\hspace{\parindent} Every game must be derived from \texttt{Engine} class declared inside \texttt{core.h}. It contains four methods, which names explain themselves. In our opinion, We have chosen an interesting approach during designing this class, as it doesn't contain any members. It is helpful among others for not breaking ABI, and every action as initialization (\texttt{init} function) has to be consciously chosen. Besides it, \texttt{Engine} class uses default constructor and copy and move operations are not allowed (however we could use here \hyperref[sec:utils]{\ref*{sec:utils} Utils}) header already containing macros for that, but we wanted to keep the headers across the whole project as lightweight as possible.\\
\label{lst:engine_class}
\begin{lstlisting}[language=c++, caption=Engine class (./engine/include/tsengine/core.h)]
class Engine
{
public:
    Engine() = default;

    Engine(const Engine&) = delete;
    Engine& operator=(Engine&) = delete;
    Engine(const Engine&&) = delete;
    Engine& operator=(Engine&&) = delete;

    virtual bool init(const char*& gameName, unsigned& width, unsigned& height);
    virtual void loadLvL() = 0;
    virtual bool tick(const float dt) = 0;
    virtual void close();
};
\end{lstlisting}

Preferable way to start the engine is to call \texttt{TS\_MAIN} macro with the game class that is derived from \texttt{Engine} class. 
\begin{lstlisting}[language=c++, caption=Start of the example game (./game/game.cpp)]
TS_MAIN(gameClass)
\end{lstlisting}
But keep in mind that \texttt{TS\_MAIN} macro is just a simple wrapper around the \texttt{main} function that creates an instance of the class provided as the argument to the macro, then the pointer to it is passed to \texttt{run} function, and all of this inside try-catch exception block.
\begin{lstlisting}[language=c++, caption=\texttt{TS\_MAIN} definition (./engine/include/tsengine/core.h)]
#define TS_MAIN(gameClass)                        \
    int main()                                    \
    {                                             \
        try                                       \
        {                                         \
            auto result = ts::run(new gameClass); \
            if (result != ts::TS_SUCCESS)         \
            {                                     \
                return result;                    \
            }                                     \
        }                                         \
        catch (const std::exception& e)           \
        {                                         \
            std::cerr << e.what() << "\n";        \
            return EXIT_FAILURE;                  \
        }                                         \
                                                  \
        return EXIT_SUCCESS;                      \
    }
\end{lstlisting}

\label{sec:run_fun}
Let's take a look at the heart of the engine, namely, run function, declared inside \texttt{core.h} alongside with \hyperref[sec:engine_class]{\texttt{Engine} class declaration}.
\begin{lstlisting}[language=c++, caption=Run function (./engine/include/tsengine/core.h)]
int run(Engine* const engine);
\end{lstlisting}

For simplicity, in this code snippet we will avoid discussing the main loop that is the part of \texttt{run} function, but we will focus on it in the code snippet after the next one.

The most important thing that the \texttt{run} function does, is being a wrapper over \hyperref[lst:engine_class]{\ref*{lst:engine_class} \texttt{Engine} class}. It means that \texttt{run} function is responsible for initialization of \hyperref[sec:tester_adapter]{\ref*{sec:tester_adapter} Engine Tester Header}, compilation of shaders (\hyperref[sec:shader_compiler]{\ref*{sec:shader_compiler} Shader Compiler}), initialization of ECS (\hyperref[sec:code_ecs]{\ref*{sec:code_ecs} Entity Component System}), \hyperref[context]{\ref*{context} \texttt{Context}}, \hyperref[window]{\ref*{window} \texttt{Window}}, and, first and foremost, the initialization of \hyperref[sec:renderer]{\ref*{sec:renderer} Renderer}. Eventually it deinitializes all the before mentioned environment after leaving \hyperref[main_loop]{\ref*{main_loop} the main loop}. All of this secured by \hyperref[fallbacks]{\ref*{fallbacks} fallback macro}.
\begin{lstlisting}[language=c++, caption=Engine's run function (./engine/src/core/core.cpp)]
int run(Engine* const game) try
{
    std::lock_guard<std::mutex> _{engineInit};

    if (!game)
    {
        TS_ERR("Game pointer is invalid");
    }

    if (isAlreadyInitiated)
    {
        TS_ERR("Game is already initialized");
    }
    isAlreadyInitiated = true;

#ifdef TESTER_ADAPTER
    const auto testerAdapter = dynamic_cast<TesterEngine*>(game);
    bool isRenderingStarted{};
#endif

    unsigned width{1280}, height{720};
    const char* gameName = nullptr;
    if (!game->init(gameName, width, height))
    {
        TS_ERR("Game initialization unsuccessful");
    }

    if (gameName == nullptr)
    {
        gameName = defaultGameName.data();
        TS_WARN(("Game name wasn't set! Default game name selected: "s + gameName).c_str());
    }

    if (!std::filesystem::is_directory("assets"))
    {
        TS_ERR("Assets can not be found");
    }

    compileShaders("assets/shaders");

    auto player = gReg.createEntity();
    player.setTag("player");
    player.addComponent<ts::TransformComponent>();
    player.addComponent<ts::RigidBodyComponent>(7.f);
    
    auto grid = gReg.createEntity();
    grid.setTag("grid");
    grid.addComponent<ts::RendererComponent<PipelineType::GRID>>();

    // TODO: try to delay it
    game->loadLvL();

    Context ctx{gameName};
    ctx.createOpenXrContext().createVulkanContext();

    gReg.addSystem<AssetStore>();
    gReg.addSystem<MovementSystem>();
    gReg.addSystem<RenderSystem>(ctx.getUniformBufferOffsetAlignment());

    gReg.update();

    AssetStore::Models::load();

    auto window = Window::createWindowInstance(gameName, width, height);
    MirrorView mirrorView{ctx, window};
    mirrorView.createSurface();
    ctx.createVkDevice(mirrorView.getSurface());
    Headset headset{ctx};
    headset.init();
    Controllers controllers(ctx.getXrInstance(), headset.getXrSession());
    controllers.setupControllers();

    Renderer renderer{ctx, headset};
    renderer.createRenderer();
    mirrorView.connect(&headset, &renderer);

    TS_LOG("tsengine initialization completed successfully");

    window->show();
    [Main Loop]

    game->close();
    ctx.sync();
    isAlreadyInitiated = false;

    return EXIT_SUCCESS;
}
TS_CATCH_FALLBACK_WITH_CLEANER(runCleaner)
\end{lstlisting}

\newpage
\label{main_loop}
While developing the main loop, you must be extremely careful as performance here is very important because this code will be executed at least 16.6 times per second. At the beginning of the loop \texttt{deltaTime} variable value is calculated, and it contains the delta of the time that had passed since the last time it was calculated. Update of \hyperref[sec:code_ecs]{\ref*{sec:code_ecs} Entity Component System}'s systems, registry, and controllers. The last thing is synchronization between headset and \hyperref[sec:renderer]{\ref*{sec:renderer} Renderer}.
\begin{lstlisting}[language=c++, caption=Engine main loop (./engine/src/core/core.cpp)]
    auto loop = true;
    auto previousTime = std::chrono::high_resolution_clock::now();
    auto startTime = std::chrono::steady_clock::now();
    while (loop)
    {
        const auto nowTime = std::chrono::high_resolution_clock::now();
        const long long elapsedNanoseconds =
            std::chrono::duration_cast<std::chrono::nanoseconds>(nowTime - previousTime).count();
        static constexpr auto nanosecondsPerSecond = 1e9f;
        const auto dt = static_cast<float>(elapsedNanoseconds) / nanosecondsPerSecond;
        previousTime = nowTime;

#ifdef TESTER_ADAPTER 
        if ((testerAdapter != nullptr) && isRenderingStarted)
        {
            if (std::chrono::steady_clock::now() >= (startTime + testerAdapter->renderingDuration))
            {
                loop = false;
            }
        }
#endif

        if ((!game->tick(dt)) || headset.isExitRequested())
        {
            loop = false;
        }

        auto message = window->peekMessage();
        if (message == Window::Message::QUIT)
        {
            loop = false;
        }
        else if (message == Window::Message::RESIZE)
        {
            mirrorView.onWindowResize();
        }
        window->dispatchMessage();

        gReg.update();

        uint32_t swapchainImageIndex;
        const auto frameResult = headset.beginFrame(swapchainImageIndex);
        if (frameResult == Headset::BeginFrameResult::RENDER_FULLY)
        {
#ifdef TESTER_ADAPTER
            if (!isRenderingStarted)
            {
                isRenderingStarted = true;
            }
#endif

            controllers.sync(headset.getXrSpace(), headset.getXrFrameState().predictedDisplayTime);
            gReg.getSystem<MovementSystem>().update(dt, controllers);

            renderer.render(swapchainImageIndex);
            const auto mirrorResult = mirrorView.render(swapchainImageIndex);

            const auto isMirrorViewVisible = (mirrorResult == MirrorView::RenderResult::VISIBLE);
            renderer.submit(isMirrorViewVisible);

            if (isMirrorViewVisible)
            {
                mirrorView.present();
            }
        }

        if ((frameResult == Headset::BeginFrameResult::RENDER_FULLY) ||
            (frameResult == Headset::BeginFrameResult::RENDER_SKIP_PARTIALLY))
        {
            headset.endFrame(frameResult == Headset::BeginFrameResult::RENDER_SKIP_PARTIALLY);
        }
    }
\end{lstlisting}

\newpage
\subsubsection{Private TSEngine Core}
\hspace{\parindent} The core of the TSEngine seems to be quite poor compared to the rest of the codebase, but it is due to the size of \hyperref[sec:renderer]{\ref*{sec:renderer} Renderer} that takes most of the space in the core of the engine.\\
All the following classes are available in the corresponding files \texttt{*.cpp} and \texttt{*.h} files inside \texttt{./engine/src/core/} path]\\
\begin{itemize}
    \item Context\\
    \label{context}
    The starting point for the whole app, especially for the renderer abstraction, is \texttt{Context} class that shares the common encapsulation.\\
    The main purpose of \texttt{Context} class is creation of OpenXR and Vulkan instances. So it means that Khronos structures are initialized within this class, and usually there is no reason to re-initialize them after the creation of the app. It is used widely through the app, in structures of: \texttt{VkPhysicalDevice}, \texttt{VkDevice}, \texttt{VkQueue}, or \texttt{XrSystemId}. Also the \texttt{Context}, which sets up multisampling and uniform buffer alignment. This is not the end, as \texttt{Context} creates and sets up debug messengers.
    It's worth to point out that Context also loads Vulkan functions with the help of \hyperref[sec:vkLoader]{\ref*{sec:vkLoader} Vulkan Loader}, unfortunately we didn't have time to provide our own OpenXR loader.\\
    We see here a chance of improvement, as it can be accessible through the singleton pattern
    \footnote{The Singleton Pattern is a design pattern used in software engineering to ensure that a class has only one instance and provides a global point of access to it.}.
    \item Window\\
    \label{window}
    \texttt{Window} class that is an interface to be implemented by \hyperref[sec:os]{\ref*{sec:os} Operating System Specific}.\\
    As the \texttt{Window} class is strongly based on operating system as it is the layer of abstraction on the system's window, you may be interested in of development on other platforms for its more information can be found in \hyperref[sec:build_os]{\ref*{sec:build_os} Building Operating System Specific}.\\
    \texttt{Window} class contains common window members which are width and height of a window and the name of it. Abstract methods are: \texttt{createWindow}, \texttt{show}, \texttt{peekMessage}, \texttt{dispatchMessage}. \texttt{Message} enumeration containing handled actions is sent by a windows server. 
\begin{lstlisting}[language=c++, caption=Engine window interface (./engine/src/core/window.h)]
class Window
{
    TS_NOT_COPYABLE_AND_MOVEABLE(Window);

public:
    Window(const std::string_view windowName, const size_t width, const size_t height) :
        mWindowName{windowName.data() + " powered by tsengine"s}, mWidth{width}, mHeight{height}
    {}
    virtual ~Window();
    enum class Message
    {
#ifdef _WIN32
        RESIZE = WM_USER + 1,
#else
        RESIZE,
#error not implemented
#endif // _WIN32
        QUIT
    };

    virtual void createWindow() = 0;
    virtual void show() = 0;
    virtual Message peekMessage() = 0;
    virtual void dispatchMessage() = 0;

    static std::shared_ptr<Window> createWindowInstance(const std::string_view windowName, const size_t width, const size_t height);

    [[nodiscard]] size_t getWidth() const { return mWidth; }
    [[nodiscard]] size_t getHeight() const { return mHeight;}

protected:
    const size_t mWidth;
    const size_t mHeight;
    const std::string mWindowName;
};
\end{lstlisting}
    The most important function is static \texttt{createWindowInstance} function that chooses and creates appropriate window based on the system in use.
\begin{lstlisting}[language=c++, caption=\texttt{createWindowInstance} function (./engine/src/core/window.cpp)]
std::shared_ptr<Window> Window::createWindowInstance(const std::string_view windowName, const size_t width, const size_t height)
{
    if (isWindowAlreadyCreated)
    {
        TS_ERR("Window is already created");
    }

    std::shared_ptr<Window> window;

#ifdef _WIN32
    window = std::make_shared<Win32Window>(windowName, width, height);
#else
    #error "not implemented"
#endif // _WIN32

    window->createWindow();

    isWindowAlreadyCreated = true;

    return window;
}
\end{lstlisting}
    \item \texttt{Controllers}\\
    Controllers abstraction provides a basic layer of abstraction over OpenXR control devices, but the real logic related to them is hidden in \hyperref[movement_system]{\ref*{movement_system} MovementSystem}. \texttt{Controllers} class itself is responsible for setup of devices used for movement, creation of actions, update of those actions and getting the action states.
\begin{lstlisting}[language=c++, caption=Controllers header(./engine/src/core/controllers.h)]
class Controllers final
{
    TS_NOT_COPYABLE_AND_MOVEABLE(Controllers);

    static constexpr std::string_view actionSetName{"actionset"};
    static constexpr std::string_view localizedActionSetName{"Actions"};
    static constexpr std::string_view interactionProfile{"/interaction_profiles/khr/simple_controller"};

public:
    Controllers(XrInstance xrInstance, XrSession xrSession) : mInstance(xrInstance), mSession(xrSession)
    {}
    ~Controllers();

    static constexpr size_t controllerCount{2};

    void setupControllers();
    void sync(const XrSpace space, const XrTime time);

    [[nodiscard]] bool getFlyState(const size_t controllerIndex) const { return mFlyStates.at(controllerIndex); }
    [[nodiscard]] math::Mat4 getPose(const size_t controllerIndex) const { return mPoses.at(controllerIndex); }

private:
    XrInstance mInstance{};
    XrSession mSession{};
    std::array<XrSpace, controllerCount> mSpaces;
    XrActionSet mActionSet{};
    XrAction mPoseAction{}, mFlyAction{}, mTriggerAction{};
    std::array<XrPath, controllerCount> mPaths;
    std::array<math::Mat4, controllerCount> mPoses{};
    std::array<float, controllerCount> mFlyStates{};

    void createAction(
        const std::string& actionName,
        const std::string& localizedActionName,
        const XrActionType type,
        XrAction& action);

    void updateActionStatePose(const XrSession session, const XrAction action, const XrPath path, XrActionStatePose& state);
    void updateActionStateFloat(const XrSession session, const XrAction action, const XrPath path, XrActionStateFloat& state);
};
\end{lstlisting}
\end{itemize}