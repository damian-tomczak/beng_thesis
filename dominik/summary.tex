\newpage
\section{Summary}
\label{sec:summary}
\hspace{\parindent}
For the project, we have implemented a virtual reality graphics engine capable of rendering 3D scenes. It is written in modern C++ and utilizes GLSL, Vulkan, and OpenXR for graphics. For all these technologies, we have created abstractions and encapsulations with high efficiency in mind during development. We are especially proud of a few features of this project. One of them is our implementation of physically based rendering. Another feature we created for the project is the grid, which allows for better user immersion in the environment. Another strength of TSEngine lies in its expandability despite the project's size, and its capability to be built on platforms other than Windows. This is achieved through the highly developed build system architecture we created for this project. We are also quite pleased with the integration of the Virtualizer Cyberith, a device capable of enhancing VR immersion. Additionally, we are proud of our math library, which we wrote ourselves to speed up calculations. A high point of our project, as a game engine, is our implementation of the Entity Component System. Although limited because we only coded an exemplary and not a fully playable game, it is still a valuable feature for any game engine and a great technique.