\newpage
\section{What is an engine?}
\hspace{\parindent}
The concept of the engine in the context of the software development, at its core, comes down to not wanting to design and write every single thing from scratch when working on a new piece of software. In the broadest strokes of the definition, the engine is a software used to transform data, but the exact way it does so, depends on the end goal it was designed for. Due to this fact, in this section, two possible types of engines will be described, game and graphic engines.

\subsection{What is a game engine?}
\hspace{\parindent}
Game engine, as the name suggest, is specialized for making games. It means that it is necessary for it to have some additional features, two of the biggest ones are handling user input and dealing with showing processed data in real time. In the modern market, there are many ready-made game engines to choose from, however one can make their own game engine if they feel the need to do so. Choosing the game engine for the project is an important part of the game development, as this choice can shape much of the working process. While smaller developers may opt for game engines with free license or software with some royalties after reaching certain level of income, most of big game companies invest in developing their own game engine, or buying one so as not to pay royalties. Game engines can be quite compact software like for example Godot, or heavy, adapted to photorealistic graphics like Unreal Engine or CryEngine.

\subsection{What is a graphics engine?}
\hspace{\parindent}
A graphics engine is a software component that takes 3D or 2D models, textures, and other data and renders them to a 2D image. It is responsible for all aspects of the rendering process, from calculating the positions and orientations of objects to applying textures and lighting effects. Nowadays, the border between the game and graphics engines is really thin, because in reality most game engines are also graphics engines. It doesn't mean that all graphic engines are game engines, though, as graphic engines have always been used to render digital images. They have been widely used in modelling, rendering photorealistic images or simulating different environments, often with association with physics engines. 
