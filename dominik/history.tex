\newpage
\section{History of Game Engines}
\label{sec:history_game_engines}
\hspace{\parindent}
Game engines have become really advanced nowadays, but it was not always the case. Their history have begun in the 1970s, when the first game engines were relatively simple, as they were designed for early home computers and consoles with limited processing power. These engines focused on basic rendering and game logic, and they were often custom-built for specific games.

The most important milestone in the graphic engine history was development of 3D graphics engines. One of the first commercial 3D game engines was Wolf3D Engine, an engine made by John Carmack for the Wolfenstein 3D game developed by id Software, released in 1992. As most computers were too slow to use in real time for rendering 3D games, the engine used a technique called raycasting. The basic idea behind raycasting is to cast a ray from the eye of the viewer into the 3D scene. The ray is then intersected with the objects in the scene to determine which objects are visible to the viewer. The visible objects are then shaded and rendered to create the final image. Although Wolfenstein implemented only a very limited version of raycasting, its clever use .....

After making the first steps into 3D, id Software tried to develop better, less limiting 3D game engines. In 1993, they released the id Tech 1 also known as Doom engine as it was first used in the game Doom.
doom
quake
source
emergence of unity, unreal

\newpage
\section{History of Virtual Reality}
\label{sec:history_vr}
\hspace{\parindent}