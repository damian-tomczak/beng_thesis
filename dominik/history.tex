\newpage
\section{History of Game Engines}
\label{sec:history_game_engines}
\hspace{\parindent}
Game engines have become really advanced nowadays, but it was not always the case. Their history have begun in the 1970s, when the first game engines were relatively simple, as they were designed for early home computers and consoles with limited processing power. These engines focused on basic rendering and game logic, and they were often custom-built for specific games.

The most important milestone in the graphic engine history was development of 3D graphics engines. One of the first commercial 3D game engines, and the one that achieved much success was Wolf3D Engine, an engine made by John Carmack for the Wolfenstein 3D game developed by id Software, released in 1992. As most computers were too slow to use in real time for rendering 3D games, the engine used a technique called raycasting on a 2D map. The basic idea behind raycasting is to cast a ray from the eye of the viewer into the 2D scene. The ray is then intersected with the objects in the scene to determine which objects are visible to the viewer. The visible objects are then shaded and rendered to create the final image. Wolfenstein implemented clever techniques to patch the shortcomings of computers of that time, like using pre-calculated textures for its walls and floors, animating the colors of textures to make it appear as if they were moving or using 2D sprites in a 3D environment to lessen the burden on real time calculations.

After making the first steps into 3D, id Software tried to develop better, less limiting 3D game engines. In 1993, they released the id Tech 1 also known as Doom engine as it was first used in the game Doom. It was a much more advanced tool, cleverly using sprites, implementing slopes and using maps stacked on top of each other to achieve a three-dimensional environment. The Doom engine also introduced a new sprite animation system that was more sophisticated than the one used in Wolfenstein 3D, allowing sprites to be animated with multiple frames and introducing animation blending. 

After the success of the Doom engine, id Software improved it and with the release of the Quake in 1996, the Quake engine also known as id Tech 2, had become known to the public. It was one of the first major game engines to support true 3D graphics, with polygonal models and textures. It also introduced real-time lighting, allowing for dynamic shadows and lighting, further improving the immersion. One feature that cemented it as predecessor to all modern first-person-shooters is the physics engine, allowing for more realistic interactions between objects.

In recent years, game engines have become much more sophisticated and incorporate a wide range of features, like physics simulation, advanced visual effects or animations. The game engines of today constantly receive new features and improvements, according to user feedback. Best example is to show how the game engines like Unreal Engine, Unity or CryEngine are constantly updated and have become a staple for game development, even though their first iterations were created respectively in 1998, 2005 and 2002.

\newpage
\section{History of Virtual Reality}
\label{sec:history_vr}
\hspace{\parindent}
Humanity have always been fascinated by stories of different places, cultures or adventures and to ease that craving many chose to immerse themselves in books, theatrical plays and later cinema. Virtual reality is simply the next step to allow its users to experience different simulations, immersing themselves further into them. 

It is said that the first mention of the device allowing oneself to experience virtual reality was in the 1936, in the science fiction novel "Pygmalion's Spectacles" by Stanley G. Weinbaum. In the novel, a scientist invents a device that allows people to experience simulations of different environments. 

But humanity had to wait some time for such a device to be created, with the first attempts, like Sensorama simulator developed in the 1950s, not receiving wider commercial attention. Sensorama, which was developed in the 1950s by Morton Heilig was a bulky stationary machine and a person had to sit down to be able to use it. It used the combination of images, sounds, smells, and vibrations to create an immersive experience.

A breakthrough in VR was made with the development of head-mounted displays. They were chosen as the right path to take in the development of the virtual reality, as they allowed users to move when experiencing the simulation. The first commercial HMD was the Sword of Damocles, which was developed in 1968 by Ivan Sutherland.

The 1970s and 1980s saw a number of advancements in VR technology, including the development of more affordable HMDs and more powerful computers. However, VR remained a niche technology until the early 2010s, with only few companies trying to improve it.

The development of new VR technologies, such as motion tracking and haptic feedback, has helped to make VR a more immersive and realistic experience. In 2016, the release of the Oculus Rift and HTC Vive marked a turning point for VR, as they became the first headsets affordable to regular consumers to achieve mainstream success.

Since then, VR has continued to grow in popularity, with new headsets being released regularly. VR is now being used for a variety of applications, including gaming, entertainment, education, and training.

In the future, VR is expected to grow even more with large companies like Meta or Apple investing in the development of the new headsets and VR applications, the cultural acceptance that VR has achieved and human craving to experience different realities.