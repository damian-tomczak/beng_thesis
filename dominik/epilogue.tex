\newpage
\section{Problems During The Development}
\label{sec:problems}
\hspace{\parindent}
As some people say, humans learn the most by their own mistakes, so in this section some problems encountered during development of TSEngine will be shown.
\begin{itemize}
    \item {VR on Linux}\\
    During the process of familiarizing ourselves with the headset available to the robotics club, we wasted many hours looking for OpenXR runtime and trying to make it work properly on Linux, but to no avail. Our best shot was Monado - the only still updated, open-source OpenXR runtime, but it still didn't work out for us. 
    
    \item {Difficulties of single person development when working with treadmill}\\
    Even though we strictly cooperated and often talked to each other when working on the project, we rarely got the chance to see each other in person due to personal reasons and different schedules. Because of this, working with Virtualizer proved to be quite challenging, as when developing features involving the treadmill as a single person, that person had to first write the code, compile it, then get into the Virtualizer, check if things are working properly, if not get out of the Virtualizer write changes and begin the process anew. Although the process proved to be quite difficult, it was doable, if lengthy. The worst part of single development was debugging the Virtualizer input as it required both being on the treadmill to input values and continue cycling through breakpoints, but with determination and use of desktops available in VR it has been done.
    
    \item {Steam OpenXR is prone to errors}\\
    When developing TSEngine, we were stunned by the series of errors, which occurred only on Vive devices, and only at the beginning when opening the application. After looking into the problem, we reached the conclusion that it wasn't a problem in our implementation, but an internal failure of OpenXR when paired with steam runtime. As steam runtime is close sourced, we ultimately weren't able to fix the errors, only catch and handle them in our code. 
    
    \item {Hardware crashes of Oculus Quest and outdated debug tools}\\
    Working with Oculus Quest proved to be quite problematic, too. Debugging was unnecessarily problematic, as the Oculus was constantly going into standby mode if not on the face, but fortunately, there was an option to disable the feature, unfortunately the option didn't work. Also, restarting Oculus had to be done quite often due to it crashing when debugging. Sometimes, the fastest way to continue debugging was to restart the whole pc.
    
    \item {RenderDoc refusing to work on AMD GPU}\\
    As we have been using RenderDoc in the development of TSEngine, it was quite a shock when after getting a new GPU, it turned out that RenderDoc didn't capture the framebuffers properly with this setup. It raised some difficulties, as some errors with first framebuffers occurred only on HTC Vive headset. This was resolved by debugging this part of TSEngine in the robotics club, but it caused some delays.
\end{itemize}

\newpage
\section{Division Of The Labor}
\label{sec:labor}
\hspace{\parindent}
TSEngine and thesis based on this project is a fruit of work of two persons: Damian Tomczak(contact@damian-tomczak.pl) and Dominik Słodkowski(dominikslodkowski01@gmail.com).
But the treadmill part of this project and the inspiration of giving to it virtual reality shape would be impossible without our thesis supervisor PhD Piotr Artiemjew that invited us to the robotic club working at Faculty of Mathematics and Computer Science at University of Warmia and Mazury, where virtual devices are open for robotic club's members.

As we have been strongly cooperating when working on the project, it is impossible to point out which person is responsible for each specific part of the work. Still, it is feasible to differentiate who contributed more or played the role of the supervisor for some parts of the work.

In the aspect of writing the thesis, Dominik mainly contributed to the theory part, while Damian was mostly responsible for code description, as he played the architect and graphics programmer roles while developing TSEngine. Dominik was responsible for implementing the treadmill functionality and for many aspects of the math library, his role in refactoring the tedious parts of code should also not be forgotten.

Damian's knowledge, based on his experience in game and graphics engine development, was crucial support during the process of creation of the thesis when something was not exactly clear.  

\section{Epilogue}
While developing the engine with the spirit of simultaneously game, graphics and virtual reality, we have consolidated a lot of knowledge gained during our higher education, private projects and professional experience. Development of such engine is an arduous and long-term process, therefore we have plans for the future of our project. 
\section{Plans for TSEngine's Future}
\label{sec:future}
So far we have plans for two features of TSEngine, and although they are not what want we want to develop now, they both have a strong impact on the code architecture so they need to be implemented before everything else: 
\begin{itemize}
    \item Deferred rendering\\
    Currently, most of the games have a lot of light sources in their scenes, therefore forward method of rendering is much heavier for CPU than deferred - \hyperref[sec:defer_vs_forward]{\ref*{sec:defer_vs_forward} Deferred vs Forward Rendering}
    \item Graphical User Interface\\
    Engines without GUI are the song of the past, therefore we need to  provide a way for graphical interaction with the ECS and core of the engine as soon as possible. Nowadays, popular and smart decision is to base UI on the ImGui library that is the leading tool in this aspect.  
\end{itemize}