\newpage
\section{Problems During The Development}
\label{sec:problems}
\hspace{\parindent}
\begin{itemize}
    \item linux vr sucks
    \item single development of treadmill sucks
    \item Steam OpenXR implementation sucks
    \item hardware crashes of Oculus Quest and connection problems and outdated debug tools
    \item renderdoc on amd sucks
\end{itemize}
\newpage
\section{Division Of The Labor}
\label{sec:labor}
\hspace{\parindent}
TSEngine and thesis based on this project is a fruit of work of two persons: Damian Tomczak(contact@damian-tomczak.pl) and Dominik Słodkowski(dominikslodkowski01@gmail.com).
But the treadmill part of this project and the inspiration of giving to it virtual reality shape would be impossible without our thesis supervisor PhD Piotr Artiemjew that invited us to the robotic club working at Faculty of Mathematics and Computer Science at University of Warmia and Mazury, where virtual devices are open for robotic club's members.

As we have been working strongly close to each other, it is impossible to point out one person is responsible only for specific part of the work. However, still we can say who more contributed to some of the parts of the work or rather who played the role of the guardian for the specific parts.\\
In the aspect of thesis Dominik's contribution played the biggest role in theory part, on the other hand Damian was mostly responsible for code description, as he played the architect role while developing TSEngine and graphics programmer. Dominik was responsible to implement the treadmill working in the project and in many aspects of the math library, should also not be forgotten his role in refactoring the tedious parts of code.\\ Damian's knowledge based on his experience in game and graphics engines was playing  undervalued support during creation of the thesis when something was not exactly clear.  
\section{Epilogue}
\section{Plans for TSEngine Future}
\begin{itemize}
    \item Deferred rendering
    \item ui
\end{itemize}