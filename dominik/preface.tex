\newpage
\section{Thesis Conventions}
Every path mentioned in the thesis starts from the main path of project repository.\\
We have two methods, the separation of source code and code names from the text:
\begin{itemize}
    \item Inline code snippet:\\
    \texttt{puts("Hello World!");}
    \item Code listening:
\begin{lstlisting}[language=c++, caption=Example code snippet (./example\_dir/example\_file.cpp)]
std::cout << "Hello World!\n";
\end{lstlisting}
\end{itemize}
The paper is consisted of three distinguish main parts, theory starting from \hyperref[sec:preface]{\ref*{sec:preface} Preface}, code description of TSEngine \hyperref[sec:code_descr]{\ref*{sec:code_descr} Code Description}, and ending starting with \hyperref[sec:problems]{\ref*{sec:problems} Problems During The Development}.

\newpage
\section{Preface English} % TODO: dominik check if it is still up-to-date
\label{sec:preface}
The thesis focuses on the architecture and implementation of real-time graphics software for virtual reality devices. To deepen the immersion of being in a virtual environment, the software is extended to support an omnidirectional treadmill. We also cover a topic related to teamwork and software development aimed at third parties. The work aims to develop and solidify programming skills, and knowledge of three-dimensional graphics, focusing on application in fanomatics. We also consider the history, present and future of the 3D graphics industry in the context of video games, and virtual reality.

The thesis begins with a theoretical section presenting the important steps in the evolution of game technology, which is 3D real-time graphics engines, starting with Wolfenstein 3D and ending with multimedia giants like Unreal Engine 5 or Unity; covering the development of the entertainment industry in the last thirty years on the basis of console generations. Special attention is given to the strongly developing virtual reality industry, touching the subject of the early VR of the previous highly experimental age. It concludes with commercial devices, whose popularity exploded in the middle of the last decade with the emergence of a startup in the form of Oculus, to end with observations on the third generation of devices, which is the generation that begins at the time of writing the engineering thesis.

The lion's share of the thesis is oriented around a sample 3D virtual reality graphics engine, named TSEngine in honor of the thesis authors.

This is a program strongly focused on the use of modern technologies centered around obtaining maximum performance at the expense of greater effort and knowledge of the recipient of such solutions. Bearing in mind that our craftsmanship is an intermediate link on the way to the final product (as opposed to traditional software engineering, when the recipient of the application is either a non-technical person or a member of the team in which the solution is developed) and due to the size of the project, we must take create software with suitable architecture and clean code.

Guided by all this, the implementation is carried out, using a high-level programming language such as C++ as defined by the International Organization for Standardization in the document ISO/IEC 14882:2020 ultimately limiting itself to the solutions specified in the same document in the :2017 edition.

For GPU support and communication with virtual reality hardware, we support the Khronos group's multiplatform standards, for the former we use the Vulkan programming interface, while the head-mounted display along with the controllers is driven by the OpenXR API.

Due to the abysmal support for virtual reality devices in the Linux family, we have decided not to multiplatform in this context, however, we still ensure that the software can be developed on any platform through the CMake tool, which will be used to manage the software build and compilation process. Software, however, will be primarily customized and tested in a Windows 10/11 environment and optimized for MSVC compiler capabilities. We will test individual code fragments with unit tests, using the GoogleTest library. The hardware on which we test the software includes HTC VIVE Pro headset, Oculus Quest 2 and NVIDIA's RTX 3000, and 2000 series GPUs.

The engine is to support the Cyberith Virtualizer Elite 2 omnidirectional treadmill, an alternative to traditional movement with controllers. In the context of the software, the use of the treadmill involves the need to support alternative movement, considering not constant movement speed and the distinction between body direction, and direction of movement.

An important premise of the thesis is the use of the Entity Component System (ECS) architectural pattern. This pattern distinguishes three key elements:
\begin{itemize}
    \item Entity - which, by means of an identifier, represents each object in the environment.
    \item Component - which is assigned to an entity by means of a modification of its signature, defines its characteristics, and stores the data needed for use in the system.
    \item System - that is, processes called on all objects that meet a given signature.
\end{itemize}

The use of this architecture is highly recommended for the development of graphics engines for games, as they significantly facilitate the implementation of gameplay elements and optimize the CPU cache.

Another important feature of our software is physically based rendering, which provides a spectacular visual experience with relatively low computational power compared to other non-raster-based lighting methods.

In developing the software, we distinguish the problems and observations involved in creating the tool not only from the perspective of the engine developer, but also focus on the programmers, designers and artists who would benefit from such software. This involves using the right architecture on which they could superstructure their application and without creating too much of a layer of abstraction, which is the bane of current solutions on the market.

Since TSEngine is an engine designed for virtual reality, this comes with its corresponding challenges. The engine needs to display two images on the headset, which we solve at the hardware level by using the Vulkan extension VK\_KHR\_multiview. This is a simplified solution that improves the engine's performance, as it allows rendering a separate image for each eye, with little computational overhead.

Source code of this project is available at:\\
\href{https://github.com/damian-tomczak/tsengine}{https://github.com/damian-tomczak/tsengine}\\
Whereas prebuilt binaries of the project are available at (those executables were built with Release build configuration and are available in two variants with and without \hyperref[sec:hardware]{Virtualizer} support):\\
\href{https://github.com/damian-tomczak/tsengine/releases}{https://github.com/damian-tomczak/tsengine/releases}.\\
Non-compiled \LaTeX thesis is available at:\\
\href{https://github.com/damian-tomczak/beng_thesis}{https://github.com/damian-tomczak/beng\_thesis}\\
Instructions on how to build the code by yourself are detailed in \hyperref[sec:how_to_run]{"Instruction How To Build the Project"} section.

Building a graphics/game engine is a long-term process, so this dissertation does not cover the engine's final form.
On November 9, 2023, the code finally achieved MVP status. The rest of the thesis will be based on the status of the code from that day, unless otherwise stated or fixes are applied to the fragments of interest.
% TODO: add yt link
\newpage
\section{Preface Polish}  % TODO: dominik polish it heheheh
\newpage
\section{What is an engine?}
\subsection{What is a game engine?}
\subsection{What is a graphics engine?}
\newpage
\section{History of Game Engines}
\label{sec:history_game_engines}
\hspace{\parindent}
Game engines have become really advanced nowadays, but it was not always the case. Their history have begun in the 1970s, when the first game engines were relatively simple, as they were designed for early home computers and consoles with limited processing power. These engines focused on basic rendering and game logic, and they were often custom-built for specific games.

The most important milestone in the graphic engine history was development of 3D graphics engines. One of the first commercial 3D game engines, and the one that achieved much success was Wolf3D Engine, an engine made by John Carmack for the Wolfenstein 3D game developed by id Software, released in 1992. As most computers were too slow to use in real time for rendering 3D games, the engine used a technique called raycasting on a 2D map. The basic idea behind raycasting is to cast a ray from the eye of the viewer into the 2D scene. The ray is then intersected with the objects in the scene to determine which objects are visible to the viewer. The visible objects are then shaded and rendered to create the final image. Wolfenstein implemented clever techniques to patch the shortcomings of computers of that time, like using pre-calculated textures for its walls and floors, animating the colors of textures to make it appear as if they were moving or using 2D sprites in a 3D environment to lessen the burden on real time calculations.

After making the first steps into 3D, id Software tried to develop better, less limiting 3D game engines. In 1993, they released the id Tech 1 also known as Doom engine as it was first used in the game Doom. It was a much more advanced tool, cleverly using sprites, implementing slopes and using maps stacked on top of each other to achieve a three-dimensional environment. The Doom engine also introduced a new sprite animation system that was more sophisticated than the one used in Wolfenstein 3D, allowing sprites to be animated with multiple frames and introducing animation blending. 

After the success of the Doom engine, id Software improved it and with the release of the Quake in 1996, the Quake engine also known as id Tech 2, had become known to the public. It was one of the first major game engines to support true 3D graphics, with polygonal models and textures. It also introduced real-time lighting, allowing for dynamic shadows and lighting, further improving the immersion. One feature that cemented it as predecessor to all modern first-person-shooters is the physics engine, allowing for more realistic interactions between objects.

In recent years, game engines have become much more sophisticated and incorporate a wide range of features, like physics simulation, advanced visual effects or animations. The game engines of today constantly receive new features and improvements, according to user feedback. Best example is to show how the game engines like Unreal Engine, Unity or CryEngine are constantly updated and have become a staple for game development, even though their first iterations were created respectively in 1998, 2005 and 2002.

\newpage
\section{History of Virtual Reality}
\label{sec:history_vr}
\hspace{\parindent}
Humanity have always been fascinated by stories of different places, cultures or adventures and to ease that craving many chose to immerse themselves in books, theatrical plays and later cinema. Virtual reality is simply the next step to allow its users to experience different simulations, immersing themselves further into them. 

It is said that the first mention of the device allowing oneself to experience virtual reality was in the 1936, in the science fiction novel "Pygmalion's Spectacles" by Stanley G. Weinbaum. In the novel, a scientist invents a device that allows people to experience simulations of different environments. 

But humanity had to wait some time for such a device to be created, with the first attempts, like Sensorama simulator developed in the 1950s, not receiving wider commercial attention. Sensorama, which was developed in the 1950s by Morton Heilig was a bulky stationary machine and a person had to sit down to be able to use it. It used the combination of images, sounds, smells, and vibrations to create an immersive experience.

A breakthrough in VR was made with the development of head-mounted displays. They were chosen as the right path to take in the development of the virtual reality, as they allowed users to move when experiencing the simulation. The first commercial HMD was the Sword of Damocles, which was developed in 1968 by Ivan Sutherland.

The 1970s and 1980s saw a number of advancements in VR technology, including the development of more affordable HMDs and more powerful computers. However, VR remained a niche technology until the early 2010s, with only few companies trying to improve it.

The development of new VR technologies, such as motion tracking and haptic feedback, has helped to make VR a more immersive and realistic experience. In 2016, the release of the Oculus Rift and HTC Vive marked a turning point for VR, as they became the first headsets affordable to regular consumers to achieve mainstream success.

Since then, VR has continued to grow in popularity, with new headsets being released regularly. VR is now being used for a variety of applications, including gaming, entertainment, education, and training.

In the future, VR is expected to grow even more with large companies like Meta or Apple investing in the development of the new headsets and VR applications, the cultural acceptance that VR has achieved and human craving to experience different realities.
\newpage
\section{Interesting Techniques} % TODO: not only in our engine
\subsection{Interesting Engine Techniques}
\subsubsection{Entity Component System}
\label{sec:theory_ecs}
\subsubsection{Reflection System}
\label{sec:refl}

\newpage
\subsection{Interesting Rendering Techniques}
\subsubsection{Differed vs Forward Rendering}

\newpage
\section{Creation of 3D World Immersion}
\[MVP\]
\[MV[2]P[2]\]

\[
\begin{bmatrix}
Orthographic\\
Projection \\
Matrix
\end{bmatrix} 
*
\begin{bmatrix}
X\\
Y\\
Z\\
1\\
\end{bmatrix} 
\]

\[
\begin{bmatrix}
Perspective\\
Projection \\
Matrix
\end{bmatrix} 
*
\begin{bmatrix}
X\\
Y\\
Z\\
1\\
\end{bmatrix} 
\]

\[
a=\frac{h}{w}
\]

\[
f=\frac{1}{tan(\theta / 2)}
\]

\[
\lambda=\frac{zfar}{zfar - znear)}
-
\frac{zfar * znear}{zfar-znear}
\]

\[
\begin{bmatrix}
x\\
y\\
z\\
\end{bmatrix}
\]
Conversion to screen space
\[
\begin{bmatrix}
afx\\
fy\\
\lambda z-\lambda znear\\
\end{bmatrix}
\]
Perspective divide
\[
x/z\ y/z\ z/z
\]

\[
\begin{bmatrix}
(\frac{h}{w})(\frac{1}{tan(\theta/2)}) & 0 & 0 & 0\\
0 & (\frac{1}{tan(\theta/2)}) & 0 & 0\\
0 & 0 & \frac{zfar}{zfar - znear)} & -\frac{zfar * znear}{zfar-znear}\\
0 & 0 & 1 & 0
\end{bmatrix} 
*
\begin{bmatrix}
X\\
Y\\
Z\\
1\\
\end{bmatrix} 
\]

An example of view matrix with camera placed at: 10 5 10:
\[
\begin{bmatrix}
1 & 0 & 0 & -10\\
0 & 1 & 0 & -5\\
0 & 0 & 1 & -10\\
0 & 0 & 0 & 1\\
\end{bmatrix} 
\]

\newpage
\section{Software Testing}
\label{sec:testing}
\hspace{\parindent}
Software development can be a lengthy and convoluted process, so to assure the quality of the project and to prevent hidden errors from disturbing workflow, it is necessary to test the code. There are various methodologies used in the process of software testing, and it is of great importance to be aware of them and best situations to use each of them.
\subsection{Unit Tests}
\hspace{\parindent}
One of the simplest ways for programmers to test their code in development is to use unit tests. This method of software testing focuses on validating the functionality of a single unit or component of the tested code. Usually, unit test are written by the same person or team that coded the feature right after finishing said feature. The main advantage of using unit tests is the ease of implementation and high ratio of finding bugs that are quite obvious but potentially very dangerous for future production. As this type of testing was used in the development of TSEngine, the exact way it was used is described in \hyperref[sec:tests]{\ref*{sec:tests} Tests} section.

\subsection{Smoke Tests}
\hspace{\parindent}
Smoke testing, also known as sanity testing or build verification testing, is a methodology that focuses on quickly checking whether the software has any obvious errors or whether the most basic functionalities work properly. This type of testing is the most suited to determine if the software is ready for more in depth tests, or if it is ready for development of the new features. Although this way of testing was used in the development of TSEngine, there is not any documentation of the process, as it was always a simple check after implementing a new feature.

\newpage
\section{Teamwork}
\label{sec:teamwork}
\hspace{\parindent} % TODO: place here any image to make it more funny
TSEngine and thesis based on this project is a fruit of work of two persons: Damian Tomczak(contact@damian-tomczak.pl) and Dominik Słodkowski(dominikslodkowski01@gmail.com).
But the treadmill part of this project and the inspiration of giving to it virtual reality shape would be impossible without our thesis supervisor PhD Piotr Artiemjew that invited us to the robotic club working at Faculty of Mathematics and Computer Science at University of Warmia and Mazury, where virtual devices are open for robotic club's members.

As in all projects that involve multiple people, it is crucial to develop a way to share each person's work and to have access to previous iterations of the software. Nowadays, it is of great ease with the emergence of version control systems such as Git or Perforce. 

The TSEngine uses Git and furthermore GitHub to accommodate this problem, as described in detail in \hyperref[sec:scv]{\ref*{sec:scv} Source Control Version} section. Git is released under the open source license -  GNU General Public License version 2.0, and as an open source software is widely used in many sites and platforms: GitHub, GitLab or Bitbucket to name the most popular ones, but there are many more developed and used to accommodate specific needs and preferences of the developers.

Another popular method of version control, although not used in the development of TSEngine, is Perforce, now rebranded as Helix Core. It is a powerful tool used mainly in larger projects by AAA game developers, visual effects studios and semiconductor companies. When using Perforce, one must note that it is proprietary software of Perforce Software, Inc. and is free to use only in teams up to 5 users and 20 workspaces. Perforce is more focused on changing files and is better suited than git when dealing with binary files. As always, in software development it is important to choose the correct tool for the project and it can make the process of development much easier



